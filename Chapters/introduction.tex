%!TEX root = ../template.tex
%%%%%%%%%%%%%%%%%%%%%%%%%%%%%%%%%%%%%%%%%%%%%%%%%%%%%%%%%%%%%%%%%%%
%% chapter1.tex
%% NOVA thesis document file
%%
%% Chapter with introduction
%%%%%%%%%%%%%%%%%%%%%%%%%%%%%%%%%%%%%%%%%%%%%%%%%%%%%%%%%%%%%%%%%%%

\typeout{NT FILE introduction.tex}%

\chapter{Introduction}
\label{cha:introduction}


% epigraph configuration
\epigraphfontsize{\small\itshape}
\setlength\epigraphwidth{12.5cm}
\setlength\epigraphrule{0pt}

\epigraph{
  Everything must be made as simple as possible. But not simpler. (Einstein)
}

\section{Context and problem definition}
\label{sec:if_you_use_this_template} 

\subsection{Overview of foodtech industry in Brazil}
\label{sub:foodtech_in_brazil}

In recent years, the food delivery market has significantly altered global consumption habits.
The advent of technology has been pivotal in this shift, enabling businesses to enhance customer experiences through
rapid delivery times, an extensive selection of food options, and competitive pricing.

Within this global trend, Brazil has emerged as a notable force in the food delivery sector, experiencing remarkable
growth, particularly amid the COVID-19 pandemic. This surge can be attributed to an increase in user base and order
frequency. A glimpse into the evolving food delivery landscape in Brazil can be found in \parencite{hortaDigitalFoodEnvironment2021}


\subsection{iFood's growth strategy based on intelligent vouchers incentives}
\label{sub:the_problem}

iFood stands at the forefront of this market in Brazil, reportedly handling over 80 million orders monthly
from more than 20 million users {\color{red}(TO CONFIRM NUMBERS)}. This positions iFood not only as a market leader but also
as a key influencer in shaping consumer behavior within the country. Other currently important factor is the challenging
macroeconomic environment, which pushes iFood to continue prioritizing investment efficiency to sustain its growth trajectory.

Such a scenario is reflected in the company's growth strategy, which is heavily based on the use of intelligent vouchers campaings
to incentivize purchases. In the face of market demands for profitability, the company is honing its investment strategies to maximize 
the effectiveness of these incentives, hence having a better planning and strategy to operate its vouchers campaings is crucial.

A nuanced understanding of customer responses to voucher promotions is essential for optimizing financial outcomes and 
fostering growth within vouchers initiatives. Specifically, the goal is to target customers whose purchasing decisions are directly 
influenced by vouchers, thereby minimizing cannibalization —where vouchers merely shift existing purchase intentions rather
than generating new ones.

\subsection{Problem statement}
\label{sub:the_problem}

While mitigating cannibalization addresses short-term goals, comprehending the long-term impacts of voucher strategies is 
essential for sustainable growth. The challenge lies in ensuring that customers incentivized by vouchers continue to engage
with the platform without becoming reliant on discounts. This balance is vital for fostering long-term customer loyalty and
maintaining a robust business model.

The dynamics of Brazil's food delivery market, exemplified by iFood's strategies, illustrate the complex interplay 
between technological innovation, consumer behavior, and strategic marketing in the digital age. The sector's evolution 
underscores the necessity of adaptive strategies that not only attract immediate sales but also build enduring customer
relationships.

\section{Research gap and objectives}
\label{sub:research_gap_and_objectives}

This research aims to address the gap in understanding the long-term impacts of voucher incentives on customer behavior, 
specially regarding the incremental purchases. Another important consideration is to understand how differently the customer
base reacts to those incentives, allowing a better targeting to tailored marketing strategies.

The primary objective is to illuminate the long-term, customer-specific reactions to voucher incentives. This insight
is pivotal for optimizing iFood's marketing strategies, ensuring that promotional expenses not only drive immediate 
sales but also contribute to sustainable, profitable growth.

\section{Mothodology used}
\label{sub:methodoly}

Estimating causal impacts, particularly those that are heterogeneous and manifest over the long term, presents a 
significant challenge. To accurately assess these effects, this study will employ a fusion of machine learning 
algorithms and causal inference techniques, applied to experimental data gathered from a six-month initiative 
conducted by iFood.

In the following section, we delve into the complexities surrounding causal inference. We aim to offer a comprehensive
overview of the methodologies, alongside their limitations and the assumptions necessary for executing robust modeling
and inferences.

\section{Thesis structure}

The remainder of this thesis is structured as follows:

\begin{itemize}
  \item \textbf{Chapter 2} offers an in-depth literature review, analyzing existing studies on voucher incentives and
  introducing key causal inference methodologies, along with their primary assumptions and limitations. This chapter 
  aims to lay the groundwork for understanding the challenges faced and the methods applied in tackling the research 
  question. It ends by advancing the conversation on causal inference, delineating the distinctions between 
  conventional machine learning models and causal inference techniques. It explores how integrating the strengths 
  of both approaches can effectively address the research problem. This chapter provides a comprehensive examination 
  of uplift modeling, meta-learners, and others sophisticated frameworks designed for causal inference. It also presents
  some real cases of vouchers incentives optimization the relies on such methodologies.
  \item \textbf{Chapter 3} outlines the data collection process, detailing the experimental design related to the used data 
  and the preprocessing steps taken to ensure the data's quality and reliability.
  \item \textbf{Chapter 4} discusses the results of the analysis, offering insights into the long-term impacts of 
  voucher incentives on customer behavior.
  \item \textbf{Chapter 5} concludes the thesis, summarizing the findings and offering recommendations for iFood's 
  marketing strategies.

\end{itemize}


